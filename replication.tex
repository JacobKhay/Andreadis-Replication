% Options for packages loaded elsewhere
% Options for packages loaded elsewhere
\PassOptionsToPackage{unicode}{hyperref}
\PassOptionsToPackage{hyphens}{url}
\PassOptionsToPackage{dvipsnames,svgnames,x11names}{xcolor}
%
\documentclass[
]{article}
\usepackage{xcolor}
\usepackage{amsmath,amssymb}
\setcounter{secnumdepth}{5}
\usepackage{iftex}
\ifPDFTeX
  \usepackage[T1]{fontenc}
  \usepackage[utf8]{inputenc}
  \usepackage{textcomp} % provide euro and other symbols
\else % if luatex or xetex
  \usepackage{unicode-math} % this also loads fontspec
  \defaultfontfeatures{Scale=MatchLowercase}
  \defaultfontfeatures[\rmfamily]{Ligatures=TeX,Scale=1}
\fi
\usepackage{lmodern}
\ifPDFTeX\else
  % xetex/luatex font selection
\fi
% Use upquote if available, for straight quotes in verbatim environments
\IfFileExists{upquote.sty}{\usepackage{upquote}}{}
\IfFileExists{microtype.sty}{% use microtype if available
  \usepackage[]{microtype}
  \UseMicrotypeSet[protrusion]{basicmath} % disable protrusion for tt fonts
}{}
\makeatletter
\@ifundefined{KOMAClassName}{% if non-KOMA class
  \IfFileExists{parskip.sty}{%
    \usepackage{parskip}
  }{% else
    \setlength{\parindent}{0pt}
    \setlength{\parskip}{6pt plus 2pt minus 1pt}}
}{% if KOMA class
  \KOMAoptions{parskip=half}}
\makeatother
% Make \paragraph and \subparagraph free-standing
\makeatletter
\ifx\paragraph\undefined\else
  \let\oldparagraph\paragraph
  \renewcommand{\paragraph}{
    \@ifstar
      \xxxParagraphStar
      \xxxParagraphNoStar
  }
  \newcommand{\xxxParagraphStar}[1]{\oldparagraph*{#1}\mbox{}}
  \newcommand{\xxxParagraphNoStar}[1]{\oldparagraph{#1}\mbox{}}
\fi
\ifx\subparagraph\undefined\else
  \let\oldsubparagraph\subparagraph
  \renewcommand{\subparagraph}{
    \@ifstar
      \xxxSubParagraphStar
      \xxxSubParagraphNoStar
  }
  \newcommand{\xxxSubParagraphStar}[1]{\oldsubparagraph*{#1}\mbox{}}
  \newcommand{\xxxSubParagraphNoStar}[1]{\oldsubparagraph{#1}\mbox{}}
\fi
\makeatother


\usepackage{longtable,booktabs,array}
\usepackage{calc} % for calculating minipage widths
% Correct order of tables after \paragraph or \subparagraph
\usepackage{etoolbox}
\makeatletter
\patchcmd\longtable{\par}{\if@noskipsec\mbox{}\fi\par}{}{}
\makeatother
% Allow footnotes in longtable head/foot
\IfFileExists{footnotehyper.sty}{\usepackage{footnotehyper}}{\usepackage{footnote}}
\makesavenoteenv{longtable}
\usepackage{graphicx}
\makeatletter
\newsavebox\pandoc@box
\newcommand*\pandocbounded[1]{% scales image to fit in text height/width
  \sbox\pandoc@box{#1}%
  \Gscale@div\@tempa{\textheight}{\dimexpr\ht\pandoc@box+\dp\pandoc@box\relax}%
  \Gscale@div\@tempb{\linewidth}{\wd\pandoc@box}%
  \ifdim\@tempb\p@<\@tempa\p@\let\@tempa\@tempb\fi% select the smaller of both
  \ifdim\@tempa\p@<\p@\scalebox{\@tempa}{\usebox\pandoc@box}%
  \else\usebox{\pandoc@box}%
  \fi%
}
% Set default figure placement to htbp
\def\fps@figure{htbp}
\makeatother


% definitions for citeproc citations
\NewDocumentCommand\citeproctext{}{}
\NewDocumentCommand\citeproc{mm}{%
  \begingroup\def\citeproctext{#2}\cite{#1}\endgroup}
\makeatletter
 % allow citations to break across lines
 \let\@cite@ofmt\@firstofone
 % avoid brackets around text for \cite:
 \def\@biblabel#1{}
 \def\@cite#1#2{{#1\if@tempswa , #2\fi}}
\makeatother
\newlength{\cslhangindent}
\setlength{\cslhangindent}{1.5em}
\newlength{\csllabelwidth}
\setlength{\csllabelwidth}{3em}
\newenvironment{CSLReferences}[2] % #1 hanging-indent, #2 entry-spacing
 {\begin{list}{}{%
  \setlength{\itemindent}{0pt}
  \setlength{\leftmargin}{0pt}
  \setlength{\parsep}{0pt}
  % turn on hanging indent if param 1 is 1
  \ifodd #1
   \setlength{\leftmargin}{\cslhangindent}
   \setlength{\itemindent}{-1\cslhangindent}
  \fi
  % set entry spacing
  \setlength{\itemsep}{#2\baselineskip}}}
 {\end{list}}
\usepackage{calc}
\newcommand{\CSLBlock}[1]{\hfill\break\parbox[t]{\linewidth}{\strut\ignorespaces#1\strut}}
\newcommand{\CSLLeftMargin}[1]{\parbox[t]{\csllabelwidth}{\strut#1\strut}}
\newcommand{\CSLRightInline}[1]{\parbox[t]{\linewidth - \csllabelwidth}{\strut#1\strut}}
\newcommand{\CSLIndent}[1]{\hspace{\cslhangindent}#1}



\setlength{\emergencystretch}{3em} % prevent overfull lines

\providecommand{\tightlist}{%
  \setlength{\itemsep}{0pt}\setlength{\parskip}{0pt}}



 


\usepackage{booktabs}
\usepackage{longtable}
\usepackage{array}
\usepackage{multirow}
\usepackage{wrapfig}
\usepackage{float}
\usepackage{colortbl}
\usepackage{pdflscape}
\usepackage{tabu}
\usepackage{threeparttable}
\usepackage{threeparttablex}
\usepackage[normalem]{ulem}
\usepackage{makecell}
\usepackage{xcolor}
\usepackage{siunitx}

    \newcolumntype{d}{S[
      table-align-text-before=false,
      table-align-text-after=false,
      input-symbols={-,\*+()}
    ]}
  
\usepackage{float}
% reset numbering after abstract
\usepackage{etoolbox}
\AtBeginEnvironment{abstract}{\setcounter{section}{0}}
\makeatletter
\@ifpackageloaded{caption}{}{\usepackage{caption}}
\AtBeginDocument{%
\ifdefined\contentsname
  \renewcommand*\contentsname{Table of contents}
\else
  \newcommand\contentsname{Table of contents}
\fi
\ifdefined\listfigurename
  \renewcommand*\listfigurename{List of Figures}
\else
  \newcommand\listfigurename{List of Figures}
\fi
\ifdefined\listtablename
  \renewcommand*\listtablename{List of Tables}
\else
  \newcommand\listtablename{List of Tables}
\fi
\ifdefined\figurename
  \renewcommand*\figurename{Figure}
\else
  \newcommand\figurename{Figure}
\fi
\ifdefined\tablename
  \renewcommand*\tablename{Table}
\else
  \newcommand\tablename{Table}
\fi
}
\@ifpackageloaded{float}{}{\usepackage{float}}
\floatstyle{ruled}
\@ifundefined{c@chapter}{\newfloat{codelisting}{h}{lop}}{\newfloat{codelisting}{h}{lop}[chapter]}
\floatname{codelisting}{Listing}
\newcommand*\listoflistings{\listof{codelisting}{List of Listings}}
\makeatother
\makeatletter
\makeatother
\makeatletter
\@ifpackageloaded{caption}{}{\usepackage{caption}}
\@ifpackageloaded{subcaption}{}{\usepackage{subcaption}}
\makeatother
\usepackage{bookmark}
\IfFileExists{xurl.sty}{\usepackage{xurl}}{} % add URL line breaks if available
\urlstyle{same}
\hypersetup{
  pdftitle={Local Heterogeneity in Artificial Intelligence Jobs Over Time and Space},
  pdfauthor={Jacob Khaykin; David Kane},
  colorlinks=true,
  linkcolor={blue},
  filecolor={Maroon},
  citecolor={Blue},
  urlcolor={Blue},
  pdfcreator={LaTeX via pandoc}}


\title{Local Heterogeneity in Artificial Intelligence Jobs Over Time and
Space}
\usepackage{etoolbox}
\makeatletter
\providecommand{\subtitle}[1]{% add subtitle to \maketitle
  \apptocmd{\@title}{\par {\large #1 \par}}{}{}
}
\makeatother
\subtitle{A Replication Study of Andreadis et al.~(AEA Papers and
Proceedings, 2025)}
\author{Jacob Khaykin\footnote{Solon High School,
  \href{mailto:jacobkhaykin27@solonschools.net}{\nolinkurl{jacobkhaykin27@solonschools.net}}} \and David
Kane\footnote{Institute for Globally Distributed Open Research and
  Education,
  \href{mailto:dave.kane@gmail.com}{\nolinkurl{dave.kane@gmail.com}}}}
\date{}
\begin{document}
\maketitle


\emph{JEL: J24, O33, R11}

\emph{Keywords:} Artificial Intelligence, Regional Economics, Labor
Markets

\emph{Data Availability:} The R code and data to reproduce this
replication are available in this repository:
https://github.com/JacobKhay/Andreadis-Replication.

\section*{Abstract}\label{abstract}
\addcontentsline{toc}{section}{Abstract}

We replicate Andreadis et al. (\citeproc{ref-andreadis2025}{2025}) on
the correlation between the level and growth of artificial intelligence
employment and education, innovation, and industry factors across U.S.
counties from 2014 to 2023. We successfully reproduce their main results
and extend the analysis by employing log-population weights to assess
robustness. The core associations persist, though most magnitudes
shrink. We also highlight unsupported causal claims in Andreadis et al.
(\citeproc{ref-andreadis2025}{2025}), given its observational design.

Declaration: There are no financial conflicts of interest to share.

\newpage

\section{Introduction}\label{introduction}

This paper replicates the analysis of Andreadis et al.
(\citeproc{ref-andreadis2025}{2025}) on the correlation between the
level and growth artificial intelligence employment openings and
education, innovation, and industry factors across U.S. counties from
2014 to 2023. Understanding where AI employment emerges and how it
spreads is important for both researchers and policymakers, since the
rise of AI has the potential to reshape regional economies, alter the
demand for skills, and shift patterns of innovation. County-level
variation offers a granular perspective on how these transformations
take hold across the United States.

The outcome of interest is the share of AI-related job postings at the
county-year level. This measure captures both the absolute level of AI
employment opportunities and how those opportunities evolve over time.
Tracking these dynamics provides insight into which regions gain early
access to AI-driven growth and which lag behind.

The original study links county-level AI employment to several
explanatory factors. Education is captured by the share of adults with a
college degree, innovation by local patenting activity, and industry
factors by the composition of employment across sectors. Together, these
variables represent structural characteristics that might condition
whether a region becomes a hub for AI-related work.

Andreadis et al. (\citeproc{ref-andreadis2025}{2025}) find that AI
employment is strongly correlated with higher educational attainment,
more innovation activity, and certain industry profiles. These
associations are robust across specifications, suggesting that counties
with strong human capital, innovative capacity, and aligned industries
are more likely to attract AI jobs. The results highlight structural
divides in access to AI employment opportunities across the country.

We reproduce these results and then extend the analysis by applying
log-population weights. This alternative weighting scheme reduces the
influence of the largest counties while still reflecting the relative
size of local labor markets. Under this adjustment, the core
relationships persist but the estimated magnitudes generally shrink.
This indicates that while large counties contribute to the overall
pattern, the relationships hold across the broader distribution of
counties.

Finally, we note that some of the causal interpretations in Andreadis et
al. (\citeproc{ref-andreadis2025}{2025}) are not supported by the
observational nature of the data. While the correlations are informative
and highlight important regional patterns, they do not establish that
education, innovation, or industry factors directly cause higher AI
employment. Our replication underscores the value of the evidence while
also emphasizing the limits of what can be inferred from the design.

\section{Data}\label{data}

The study utilizes multiple data sources to construct a comprehensive
county-level dataset spanning 2014--2023:

\textbf{AI Employment Data}: Job posting data from Lightcast, which
aggregates information from over 40,000 online job boards, newspapers,
and employer websites. AI-related jobs are identified through skills and
keywords associated with AI development and use. The dependent variable
\(AI_{it}\) is defined as:

\[
AI_{it} = \frac{\text{AI job postings}_{it}}{\text{Total job postings}_{it}} \times 100 \tag{1}
\]

\textbf{Demographic Variables}: From the American Community Survey
(ACS), we include bachelor's share (percentage of workforce with
bachelor's degree or higher), black population share (percentage of
county population identifying as Black), poverty share (percentage of
population below federal poverty line), log population (natural
logarithm of county population), and log median income (natural
logarithm of median household income).

\textbf{Innovation Indicators}: We measure patents per employee (USPTO
patent counts normalized by employment), AI patents share (percentage of
patents classified as AI-related), STEM degrees share (percentage of
awarded degrees in STEM fields), and degrees per capita (total degrees
awarded per capita).

\textbf{Industry and Labor Market Variables}: These include labor market
tightness (ratio of job postings to unemployed workers), manufacturing
intensity (employment share in manufacturing sector), ICT intensity
(employment share in information and communication technology), turnover
rate (worker separation rate from Quarterly Workforce Indicators), and
large establishments share (percentage of employment in large firms).

\textbf{Housing Market}: We include house price growth from Federal
Housing Finance Agency (FHFA).

All explanatory variables are lagged by one year to address potential
endogeneity concerns. The baseline specification relates the dependent
variable to these covariates using the following model:

\[
AIShare_{ct} = \alpha + \beta_1 Education_{ct} + \beta_2 Innovation_{ct} + \beta_3 Industry_{ct} + \gamma X_{ct} + \epsilon_{ct} \tag{2}
\]

where \(AIShare_{ct}\) denotes the proportion of AI-related job postings
in county \(c\) at year \(t\), and \(X_{ct}\) includes demographic,
labor market, and housing controls.

\subsection{Reproduction of Original
Results}\label{reproduction-of-original-results}

We successfully reproduced the main findings from Tables 1 and 2 of
Andreadis et al. (\citeproc{ref-andreadis2025}{2025}). The reproduction
confirms the authors' key empirical findings regarding the correlates of
AI employment across U.S. counties. All coefficients, standard errors,
and significance levels match the original results within rounding
error, demonstrating the reproducibility of their analysis.

\begin{table}[H]

\caption{\label{tbl-table1}Replication of Table 1 from Andreadis et
al.~(2025) - The Correlates of the Share of Artificial Intelligence
Jobs}

\centering{

\centering\centering
\resizebox{\ifdim\width>\linewidth\linewidth\else\width\fi}{!}{
\fontsize{18}{20}\selectfont
\begin{threeparttable}
\begin{tabular}[t]{>{\raggedright\arraybackslash}p{6.5cm}>{\centering\arraybackslash}p{2.4cm}>{\centering\arraybackslash}p{2.4cm}>{\centering\arraybackslash}p{2.4cm}>{\centering\arraybackslash}p{2.4cm}>{\centering\arraybackslash}p{2.4cm}}
\toprule
\multicolumn{1}{c}{\bgroup\fontsize{18}{20}\selectfont \textbf{ }\egroup{}} & \multicolumn{1}{c}{\bgroup\fontsize{18}{20}\selectfont \textbf{Demographics}\egroup{}} & \multicolumn{1}{c}{\bgroup\fontsize{18}{20}\selectfont \textbf{Innovation}\egroup{}} & \multicolumn{1}{c}{\bgroup\fontsize{18}{20}\selectfont \textbf{Industry}\egroup{}} & \multicolumn{1}{c}{\bgroup\fontsize{18}{20}\selectfont \textbf{All Controls}\egroup{}} & \multicolumn{1}{c}{\bgroup\fontsize{18}{20}\selectfont \textbf{All + State FE}\egroup{}} \\
\cmidrule(l{3pt}r{3pt}){2-2} \cmidrule(l{3pt}r{3pt}){3-3} \cmidrule(l{3pt}r{3pt}){4-4} \cmidrule(l{3pt}r{3pt}){5-5} \cmidrule(l{3pt}r{3pt}){6-6}
  & (1) & (2) & (3) & (4) & (5)\\
\midrule
\huge{Bachelor's share, z-score} & \num{0.160}*** &  &  & \num{0.188}*** & \num{0.085}\\
 & (\num{0.047}) &  &  & (\num{0.067}) & (\num{0.059})\\
\huge{Black pop, z-score} & \num{-0.121} &  &  & \num{-0.121} & \num{0.000}\\
 & (\num{0.127}) &  &  & (\num{0.185}) & (\num{0.162})\\
\huge{Poverty share, z-score} & \num{0.044} &  &  & \num{0.067}* & \num{-0.003}\\
 & (\num{0.028}) &  &  & (\num{0.040}) & (\num{0.037})\\
\huge{log(Population), z-score} & \num{0.807}** &  &  & \num{0.219} & \num{0.146}\\
\addlinespace[1em]
 & (\num{0.367}) &  &  & (\num{0.444}) & (\num{0.544})\\
\huge{House Price Growth, z-score} & \num{-0.019}** &  &  & \num{-0.028}** & \num{-0.032}***\\
 & (\num{0.008}) &  &  & (\num{0.012}) & (\num{0.011})\\
\huge{log(Median Income), z-score} & \num{-0.022} &  &  & \num{-0.001} & \num{0.032}\\
\addlinespace[1em]
 & (\num{0.047}) &  &  & (\num{0.065}) & (\num{0.060})\\
\huge{Labor Market Tightness, z-score} & \num{0.255}*** &  &  & \num{0.315}*** & \num{0.372}***\\
 & (\num{0.050}) &  &  & (\num{0.063}) & (\num{0.065})\\
\huge{Patents per employee, z-score} &  & \num{0.031}*** &  & \num{0.028}** & \num{0.031}**\\
 &  & (\num{0.012}) &  & (\num{0.012}) & (\num{0.014})\\
\huge{AI patents' share, z-score} &  & \num{0.009} &  & \num{0.009} & \num{0.003}\\
 &  & (\num{0.007}) &  & (\num{0.006}) & (\num{0.005})\\
\huge{Degrees awarded per capita, z-score} &  & \num{0.013} &  & \num{0.034} & \num{0.034}\\
 &  & (\num{0.026}) &  & (\num{0.026}) & (\num{0.027})\\
\huge{STEM Degrees' share, z-score} &  & \num{0.072}*** &  & \num{0.058}*** & \num{0.048}**\\
 &  & (\num{0.023}) &  & (\num{0.022}) & (\num{0.020})\\
\huge{Large Establishments, z-score} &  &  & \num{-0.002} & \num{-0.037} & \num{-0.036}\\
 &  &  & (\num{0.025}) & (\num{0.028}) & (\num{0.029})\\
\huge{ICT sector Intensity, z-score} &  &  & \num{0.010} & \num{0.038}** & \num{0.040}**\\
 &  &  & (\num{0.014}) & (\num{0.017}) & (\num{0.016})\\
\huge{Manufacturing Intensity, z-score} &  &  & \num{-0.057}*** & \num{-0.048}*** & \num{-0.034}**\\
 &  &  & (\num{0.012}) & (\num{0.014}) & (\num{0.015})\\
\huge{Turnover Rate, z-score} &  &  & \num{0.034}*** & \num{0.016} & \num{0.010}\\
 &  &  & (\num{0.011}) & (\num{0.017}) & (\num{0.016})\\
\textit{Fixed-effects} &  &  &  &  & \\
Year & Yes & Yes & Yes & Yes & Yes\\
County & Yes & Yes & Yes & Yes & Yes\\
State Year &  &  &  &  & Yes\\
\textit{Fit statistics} &  &  &  &  & \\
Observations & 27,497 & 22,744 & 24,651 & 19,184 & 19,184\\
R² & 0.704 & 0.721 & 0.694 & 0.758 & 0.778\\
Within R² & 0.052 & 0.003 & 0.002 & 0.086 & 0.095\\
\bottomrule
\multicolumn{6}{l}{\rule{0pt}{1em}* p $<$ 0.1, ** p $<$ 0.05, *** p $<$ 0.01}\\
\end{tabular}
\begin{tablenotes}[para]
\item Notes: Sources: Lightcast, American Community Survey, Quarterly Workforce Indicators, 2014-2023. The table reports coefficients from regressions of the share of AI jobs in a county on Demographic, Innovation, and Industry Characteristics. Observations are weighted by log(1 + job postings) and standard errors are clustered at the county-level. *** p<0.01, ** p<0.05, * p<0.10
\end{tablenotes}
\end{threeparttable}}

}

\end{table}%

\begin{table}[H]

\caption{\label{tbl-table2}Replication of Table 2 from Andreadis et
al.~(2025) - The Correlates of the Percentage Point Change in the Share
of AI Jobs}

\centering{

\centering\centering
\resizebox{\ifdim\width>\linewidth\linewidth\else\width\fi}{!}{
\fontsize{18}{20}\selectfont
\begin{threeparttable}
\begin{tabular}[t]{>{\raggedright\arraybackslash}p{6.5cm}>{\centering\arraybackslash}p{2.4cm}>{\centering\arraybackslash}p{2.4cm}>{\centering\arraybackslash}p{2.4cm}>{\centering\arraybackslash}p{2.4cm}>{\centering\arraybackslash}p{2.4cm}}
\toprule
\multicolumn{1}{c}{\bgroup\fontsize{18}{20}\selectfont \textbf{ }\egroup{}} & \multicolumn{1}{c}{\bgroup\fontsize{18}{20}\selectfont \textbf{Demographics}\egroup{}} & \multicolumn{1}{c}{\bgroup\fontsize{18}{20}\selectfont \textbf{Innovation}\egroup{}} & \multicolumn{1}{c}{\bgroup\fontsize{18}{20}\selectfont \textbf{Industry}\egroup{}} & \multicolumn{1}{c}{\bgroup\fontsize{18}{20}\selectfont \textbf{All Controls}\egroup{}} & \multicolumn{1}{c}{\bgroup\fontsize{18}{20}\selectfont \textbf{All + State FE}\egroup{}} \\
\cmidrule(l{3pt}r{3pt}){2-2} \cmidrule(l{3pt}r{3pt}){3-3} \cmidrule(l{3pt}r{3pt}){4-4} \cmidrule(l{3pt}r{3pt}){5-5} \cmidrule(l{3pt}r{3pt}){6-6}
  & (1) & (2) & (3) & (4) & (5)\\
\midrule
\huge{Bachelors, \% z-score in 2017} & \num{0.007} &  &  & \num{-0.069} & \num{-0.136}***\\
 & (\num{0.028}) &  &  & (\num{0.051}) & (\num{0.043})\\
\huge{Black, \% z-score in 2017} & \num{0.018} &  &  & \num{0.045} & \num{0.053}\\
 & (\num{0.020}) &  &  & (\num{0.033}) & (\num{0.034})\\
\huge{Poverty, \% z-score in 2017} & \num{0.064}* &  &  & \num{0.050} & \num{0.104}\\
 & (\num{0.037}) &  &  & (\num{0.070}) & (\num{0.075})\\
\huge{Pop. Growth} & \num{-0.016} &  &  & \num{-0.007} & \num{-0.013}\\
\addlinespace[1em]
 & (\num{0.023}) &  &  & (\num{0.057}) & (\num{0.057})\\
\huge{House Price Growth z-score in 2017} & \num{-0.032}* &  &  & \num{-0.036} & \num{0.045}\\
 & (\num{0.017}) &  &  & (\num{0.037}) & (\num{0.042})\\
\huge{Income, Log z-score in 2017} & \num{0.124}*** &  &  & \num{0.123} & \num{0.195}**\\
\addlinespace[1em]
 & (\num{0.043}) &  &  & (\num{0.077}) & (\num{0.076})\\
\huge{Tightness, z-score in 2017} & \num{0.089}*** &  &  & \num{0.178}*** & \num{0.139}\\
 & (\num{0.021}) &  &  & (\num{0.042}) & (\num{0.096})\\
\huge{Patents per employee z-score in 2017} &  & \num{0.008} &  & \num{-0.056} & \num{-0.001}\\
 &  & (\num{0.029}) &  & (\num{0.037}) & (\num{0.043})\\
\huge{AI Patents' Share z-score in 2017} &  & \num{0.137}*** &  & \num{0.108}** & \num{0.079}\\
 &  & (\num{0.036}) &  & (\num{0.043}) & (\num{0.066})\\
\huge{Degrees awarded per capita, z-score in 2017} &  & \num{0.020} &  & \num{0.048}* & \num{0.062}\\
 &  & (\num{0.016}) &  & (\num{0.028}) & (\num{0.037})\\
\huge{STEM Degrees' share, z-score in 2017} &  & \num{0.094}*** &  & \num{0.083}*** & \num{0.078}**\\
 &  & (\num{0.027}) &  & (\num{0.027}) & (\num{0.036})\\
\huge{Large Establishments, \% z-score in 2017} &  &  & \num{0.028} & \num{-0.088}* & \num{-0.039}\\
 &  &  & (\num{0.021}) & (\num{0.050}) & (\num{0.063})\\
\huge{ICT sector Intensity, \% z-score in 2017} &  &  & \num{0.027} & \num{-0.023} & \num{-0.015}\\
 &  &  & (\num{0.018}) & (\num{0.031}) & (\num{0.026})\\
\huge{Manufacturing Intensity, \% z-score in 2017} &  &  & \num{-0.034}* & \num{-0.012} & \num{-0.001}\\
 &  &  & (\num{0.019}) & (\num{0.046}) & (\num{0.043})\\
\huge{Turnover Rate, \% z-score in 2017} &  &  & \num{-0.073}*** & \num{-0.129}** & \num{-0.173}**\\
 &  &  & (\num{0.022}) & (\num{0.061}) & (\num{0.081})\\
\textit{Fixed-effects} &  &  &  &  & \\
State &  &  &  &  & Yes\\
\textit{Fit statistics} &  &  &  &  & \\
Observations & 2,751 & 897 & 2,723 & 810 & 810\\
R² & 0.025 & 0.039 & 0.009 & 0.075 & 0.187\\
Within R² & — & — & — & — & 0.065\\
\bottomrule
\multicolumn{6}{l}{\rule{0pt}{1em}* p $<$ 0.1, ** p $<$ 0.05, *** p $<$ 0.01}\\
\end{tabular}
\begin{tablenotes}[para]
\item Notes: Sources: Lightcast, American Community Survey, Quarterly Workforce Indicators, 2016–2023. The table reports the coefficients associated with regressions of the change in share of AI jobs in a county from 2017–18 (average) to 2022–23 (average) on Demographic, Innovation, and Industry characteristics. Observations are unweighted. Standard errors in parentheses. *** p<0.01, ** p<0.05, * p<0.1
\end{tablenotes}
\end{threeparttable}}

}

\end{table}%

\begin{figure}[H]

\caption{\label{fig-map}Replication of Andreadis et al.~(2025) - Spatial
heterogeneity in AI job share (Panel A, 2014--2023 average) and
percentage-point change (Panel B, 2018--2023).}

\centering{

\pandocbounded{\includegraphics[keepaspectratio]{replication_files/figure-pdf/fig-map-1.pdf}}

}

\end{figure}%

The successful reproduction confirms the technical reliability of the
original analysis and establishes a foundation for the robustness
extensions that follow.

\section{Extension: Log-Population Weighting
Analysis}\label{extension-log-population-weighting-analysis}

To assess the robustness of the original findings, we re-estimated all
models using log-population weights instead of the original weights.
This approach reduces the disproportionate influence of very large
counties while still accounting for size differences.

The modified weighting scheme is: \(w_{it} = \log(Population_{it})\)

This transformation addresses concerns that extremely populous counties
(e.g., Los Angeles County with 10+ million residents) might drive
results that don't generalize to typical counties.

\begin{figure}[H]

\caption{\label{fig-table1-coefs-models1-3}Extension of Table 1 from
Andreadis et al.~(2025) --- Coefficients under Original weights vs
Log(population) weights (Models 1-3). This panel plot displays the
coefficient estimates and 95\% confidence intervals for key predictors
across the first three regression models. Each panel represents a
different model from the original paper. Within each panel, two
estimates are shown for each variable---one using the authors' original
weights and one using log(population) weights. The associations remain
statistically significant, though most magnitudes shrink.}

\centering{

\pandocbounded{\includegraphics[keepaspectratio]{replication_files/figure-pdf/fig-table1-coefs-models1-3-1.pdf}}

}

\end{figure}%

\begin{figure}[H]

\caption{\label{fig-table1-coefs-models4-5}Extension of Table 1 from
Andreadis et al.~(2025) --- Coefficients under Original weights vs
Log(population) weights (Models 4-5). This panel plot displays the
coefficient estimates and 95\% confidence intervals for key predictors
across models 4 and 5 (All Controls and All + State FE). Each panel
represents a different model from the original paper. Within each panel,
two estimates are shown for each variable---one using the authors'
original weights and one using log(population) weights.}

\centering{

\pandocbounded{\includegraphics[keepaspectratio]{replication_files/figure-pdf/fig-table1-coefs-models4-5-1.pdf}}

}

\end{figure}%

\begin{figure}[H]

\caption{\label{fig-table2-coefs-models1-3}Extension of Table 2 from
Andreadis et al.~(2025) --- Change in AI share under Original weights vs
Log(population) weights (Models 1-3). This figure focuses on the change
in AI employment share from 2014 to 2023 for the first three models. The
variables selected represent core predictors of shifting AI employment.
Each panel reflects a different model specification, with comparisons
between original-weighted and log(population)-weighted regressions.}

\centering{

\pandocbounded{\includegraphics[keepaspectratio]{replication_files/figure-pdf/fig-table2-coefs-models1-3-1.pdf}}

}

\end{figure}%

\begin{figure}[H]

\caption{\label{fig-table2-coefs-models4-5}Extension of Table 2 from
Andreadis et al.~(2025) --- Change in AI share under Original weights vs
Log(population) weights (Models 4-5). This figure focuses on the change
in AI employment share from 2014 to 2023 for models 4 and 5 (All
Controls and All + State FE). Each panel reflects a different model
specification, with comparisons between original-weighted and
log(population)-weighted regressions.}

\centering{

\pandocbounded{\includegraphics[keepaspectratio]{replication_files/figure-pdf/fig-table2-coefs-models4-5-1.pdf}}

}

\end{figure}%

\subsection{Results}\label{results}

The comparison between equal-weighted and log-population-weighted
regressions reveals several important patterns:

\emph{Magnitude Effects}: The estimated effects of key predictors are
highly sensitive to the weighting scheme. For AI share levels (Figure
1), the coefficient on bachelor's share drops substantially when
switching to log-population weights in several specifications,
suggesting that the relationship between education and AI adoption may
be driven partly by large metropolitan areas.

\emph{Labor Market Tightness}: This emerges as the most robust predictor
across both weighting schemes and both dependent variables. In Figure 1,
labor tightness maintains strong positive effects regardless of
weighting method, and in Figure 2, it consistently predicts AI job
growth. This suggests that tight labor markets create conditions
conducive to AI adoption across counties of all sizes.

\emph{STEM Education}: STEM degree share shows consistent positive
relationships in both weighting schemes, though magnitudes vary. This
indicates that technical human capital is important for AI adoption
beyond just large metropolitan areas.

\emph{Manufacturing vs.~Technology Sectors}: Manufacturing intensity
consistently shows negative relationships with AI adoption, while ICT
intensity shows positive effects. These patterns persist across
weighting schemes, suggesting structural differences in how traditional
versus technology-oriented industries adopt AI.

\emph{County Size Effects}: The divergence between weighting schemes is
most pronounced for variables like bachelor's share and population size
itself, indicating that large counties drive many of the education--AI
relationships found in the original analysis.

\section{Conclusion}\label{conclusion}

This replication and extension of Andreadis et al.~(2025) demonstrates
both the reproducibility and the limitations of their findings. The
successful reproduction confirms that local labor market conditions,
human capital, and innovation capacity are correlated with AI employment
across U.S. counties. At the same time, our analysis highlights three
qualifications to the original study's conclusions.

\emph{First}, the original study employs causal language that overstates
what the empirical design can support. Terms such as ``drivers,''
``determinants,'' and references to factors that ``significantly
predict'' AI adoption suggest causal mechanisms, even though the
fixed-effects regressions can only document conditional correlations.
Without exogenous variation or quasi-experimental identification
strategies, these patterns likely reflect some combination of causal
effects, reverse causality, and selection processes.

\emph{Second}, the alternative log-population weighting analysis shows
that several relationships are sensitive to the influence of large
metropolitan counties. Labor market tightness remains the most
consistent predictor across both weighting schemes and both dependent
variables, suggesting that tight labor markets foster conditions
conducive to AI adoption regardless of county size. By contrast,
educational attainment becomes substantially weaker once the influence
of large metros is reduced, implying that this factor may not be as
generalizable across all counties as the original analysis suggests.

\emph{Third}, the industry composition effects prove relatively stable
across weighting schemes. Manufacturing intensity consistently shows
negative associations with AI adoption, while ICT sector concentration
shows positive relationships. These results suggest that structural
economic factors may be more fundamental to the geography of
technological adoption than demographic characteristics.

\emph{Policy implications}: Taken together, these findings indicate that
the empirical regularities documented by Andreadis et al.~(2025) should
be interpreted with caution. Policies that seek to strengthen labor
markets and industry composition appear relevant across a wide range of
counties, while education-focused interventions may generate the largest
benefits in large metropolitan areas where complementary institutions
and network effects are strongest.

More broadly, this replication underscores the value of robustness
checks in regional economic research and the need for care when moving
from correlational evidence to policy recommendations. Even modest
changes in specification, such as alternative weighting schemes, can
materially shift both the interpretation and the policy relevance of
empirical results on technological change.

\section{Critical Assessment of Causal
Claims}\label{critical-assessment-of-causal-claims}

\subsection{Identification of Problematic Causal
Language}\label{identification-of-problematic-causal-language}

The original study by Andreadis et al.~(2025) often employs language
that suggests causal relationships, even though the empirical analysis
is based on observational county-level data with multiple potentially
endogenous predictors. Several passages illustrate this concern.

In the \emph{Introduction}, the authors state:\\
\textgreater{} ``Second, we identify several key drivers of AI job
intensity, including demographics, innovation, and industry factors,
after controlling for county and year fixed effects. Specifically,
higher shares of STEM degrees, labor market tightness, and patent
activity significantly predict greater AI adoption, underscoring the
importance of education, innovation, and dynamic labor markets.''

In \emph{Section III}, they conclude:\\
\textgreater{} ``Labor market tightness emerges as a key driver, with a
positive and highly significant coefficient\ldots{} highlighting the
importance of technical education and local innovation capacity in
fostering AI job growth.''

And in the \emph{Conclusion}:\\
\textgreater{} ``Counties with stronger innovation ecosystems, higher
STEM degree attainment, and tighter labor markets have seen greater AI
job growth, whereas manufacturing-heavy regions and areas with high
labor turnover have faced challenges in integrating AI. These findings
point to the role of place-based policies to attract and retain top-tier
talent for economic development.''

Each of these statements frames correlational results as causal
mechanisms. Terms such as ``drivers,'' ``emerges as a key driver,''
``underscoring the importance,'' and ``findings point to the role of
policy'' imply that altering these variables would directly change AI
adoption outcomes. Yet the empirical strategy---fixed-effects
regressions on observational county characteristics---does not support
such causal inference. The results can only be interpreted as
conditional associations, not as estimates of the effects of education,
innovation, or labor market conditions on AI employment.

\section*{References}\label{references}
\addcontentsline{toc}{section}{References}

\phantomsection\label{refs}
\begin{CSLReferences}{1}{0}
\bibitem[\citeproctext]{ref-acemoglu2024}
Acemoglu, D. (2024). \emph{The simple macroeconomics of AI}.

\bibitem[\citeproctext]{ref-acemoglu2019}
Acemoglu, D., \& Restrepo, P. (2019). Automation and new tasks: How
technology displaces and reinstates labor. \emph{Journal of Economic
Perspectives}, \emph{33}(2), 3--30.

\bibitem[\citeproctext]{ref-aghionhowitt1992}
Aghion, P., \& Howitt, P. (1992). A model of growth through creative
destruction. \emph{Econometrica}, \emph{60}(2), 323--351.

\bibitem[\citeproctext]{ref-aghion2017}
Aghion, P., Jones, B. F., \& Jones, C. I. (2017). \emph{Artificial
intelligence and economic growth} {[}Working Paper{]}. National Bureau
of Economic Research.

\bibitem[\citeproctext]{ref-andreadis2024_muni}
Andreadis, L., Chatzikonstantinou, M., Kalotychou, E., Louca, C., \&
Makridis, C. A. (2024). \emph{The local effects of artificial
intelligence labor investments: Evidence from the municipal bond
market}. SSRN Working Paper.

\bibitem[\citeproctext]{ref-andreadis2025}
Andreadis, L., Kalotychou, E., Chatzikonstantinou, M., Louca, C., \&
Makridis, C. A. (2025). Local heterogeneity in artificial intelligence
jobs over time and space. \emph{AEA Papers and Proceedings}.

\bibitem[\citeproctext]{ref-autor2015}
Autor, D. H. (2015). Why are there still so many jobs? The history and
future of workplace automation. \emph{Journal of Economic Perspectives},
\emph{29}(3), 3--30.

\bibitem[\citeproctext]{ref-babina2024}
Babina, T., Fedyk, A., He, A., \& Hodson, J. (2024). Artificial
intelligence, firm growth, and product innovation. \emph{Journal of
Financial Economics}, \emph{151}, 103745.

\bibitem[\citeproctext]{ref-beckett2023}
Beckett, E. (2023). \emph{Demand for AI skills continues climbing}.
Lightcast Blog. \url{https://lightcast.io/}

\bibitem[\citeproctext]{ref-bogin2019}
Bogin, A., Doerner, W., \& Larson, W. (2019). Local house price
dynamics: New indices and stylized facts. \emph{Real Estate Economics},
\emph{47}(2), 365--398.

\bibitem[\citeproctext]{ref-bresnahan1995}
Bresnahan, T. F., \& Trajtenberg, M. (1995). General-purpose
technologies {``engines of growth''}? \emph{Journal of Econometrics},
\emph{61}(1), 83--108.

\bibitem[\citeproctext]{ref-brynjolfsson2021}
Brynjolfsson, E., Rock, D., \& Syverson, C. (2021). The productivity
j-curve: How intangibles complement general-purpose technologies.
\emph{American Economic Journal: Macroeconomics}, \emph{13}(1),
333--372.

\bibitem[\citeproctext]{ref-chen2019}
Chen, M., Wu, Q., \& Yang, B. (2019). How valuable is FinTech
innovation? \emph{The Review of Financial Studies}, \emph{32}(5),
2062--2106.

\bibitem[\citeproctext]{ref-eloundou2024}
Eloundou, T., Manning, S., Mishkin, P., \& Rock, D. (2024). GPTs are
GPTs: Labor market impact potential of LLMs. \emph{Science},
\emph{384}(6702), 1306--1308.

\bibitem[\citeproctext]{ref-farboodi2021}
Farboodi, M., \& Veldkamp, L. (2021). \emph{A model of the data economy}
{[}Working Paper{]}. National Bureau of Economic Research.

\bibitem[\citeproctext]{ref-giczy2022}
Giczy, A. V., Pairolero, N. A., \& Toole, A. A. (2022). Identifying
artificial intelligence (AI) invention: A novel AI patent dataset.
\emph{The Journal of Technology Transfer}, \emph{47}(2), 476--505.

\bibitem[\citeproctext]{ref-gofman2024}
Gofman, M., \& Jin, Z. (2024). Artificial intelligence, education, and
entrepreneurship. \emph{The Journal of Finance}, \emph{79}(1), 631--667.

\bibitem[\citeproctext]{ref-grennan2020}
Grennan, J., \& Michaely, R. (2020). \emph{Artificial intelligence and
high-skilled work: Evidence from analysts} (Research Paper 20-84). Swiss
Finance Institute.

\bibitem[\citeproctext]{ref-holland1986}
Holland, P. W. (1986). Statistics and causal inference. \emph{Journal of
the American Statistical Association}, \emph{81}(396), 945--960.
\url{https://doi.org/10.1080/01621459.1986.10478354}

\bibitem[\citeproctext]{ref-kline2013}
Kline, P., \& Moretti, E. (2013). People, places, and public policy:
Some simple welfare economics of local economic development programs.
\emph{Annual Review of Economics}, \emph{5}, 629--662.

\bibitem[\citeproctext]{ref-mihet2019}
Mihet, R., \& Philippon, T. (2019). The economics of big data and
artificial intelligence. In \emph{International finance review} (Vol.
20).

\bibitem[\citeproctext]{ref-romer1990}
Romer, P. M. (1990). Endogenous technological change. \emph{Journal of
Political Economy}, \emph{98}(5, Part 2), S71--S102.

\bibitem[\citeproctext]{ref-bls_laus}
U.S. Bureau of Labor Statistics. (2024). \emph{Local area unemployment
statistics (LAUS)}. \url{https://www.bls.gov/lau/}

\bibitem[\citeproctext]{ref-census_acs}
U.S. Census Bureau. (2024a). \emph{American community survey (ACS)
5-year data}.
\url{https://www.census.gov/data/datasets/time-series/econ/acs/acs-datasets.html}

\bibitem[\citeproctext]{ref-census_cbp}
U.S. Census Bureau. (2024b). \emph{County business patterns (CBP)}.
\url{https://www.census.gov/programs-surveys/cbp/data/datasets.html}

\bibitem[\citeproctext]{ref-census_qwi}
U.S. Census Bureau. (2024c). \emph{Quarterly workforce indicators
(QWI)}. \url{https://lehd.ces.census.gov/data/}

\end{CSLReferences}




\end{document}
